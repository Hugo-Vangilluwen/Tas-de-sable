\documentclass[11pt]{article}
\usepackage{amsmath, amsfonts, amssymb, amsthm, mathrsfs}
\usepackage[french]{babel}
\usepackage[utf8]{inputenc} % Caractères spéciaux
\usepackage[T1]{fontenc} % Encodage 8-bit

\theoremstyle{definition}
\newtheorem{definition}{Définition}

\title{TIPE Tas de sable}
\author{Hugo V}

\begin{document}
\maketitle

\begin{definition}
Soit $E$ un ensemble et $*: E \times E \rightarrow E$ un loi interne.
$(E, *)$ est un semi-groupe si $*$ est associative.
$(E, *)$ est un monoïde si de plus, il existe un élément neutre.
\end{definition}

\begin{definition}[Graphe de tas de sable]
Un graphe de tas de sable $G$ est un triplet $(S, A, p)$ où $(S, A)$ est un graphe orienté et $p \in S$ un puits.
On notera $\widetilde{S} = S \setminus \{p\}$ et les voisins sortant de
$s \in S$, $V^+(s) = \left\{u \in S \;|\; (s, u) \in A\right\}$.
\end{definition}

\begin{definition}[Monoïde des configurations]
Une configuration sur $G$ est une fonction
$c: \widetilde{S} \rightarrow \mathbb{N}$.
Elle indique à chaque sommet le nombre de grains de sable qu'il contient.

Les configurations munies de la loi $+$ définie par$(c + c')(s) = c(s) + c'(s)$
forment un monoïde.

Pour $s\in \widetilde{S}$, on notera $s$ la configuration valant $1$ en $s$
et $0$ partout ailleurs.
\end{definition}

\begin{definition}[Écroulement]
Pour une configuration $c$ et $s \in \widetilde{S}$, l'écroulement de $c$ en $s$
est $c' = c - \left|V^+(s)\right| s + \sum_{u \in V^+(s), u \neq p} u$.
On ne peut réaliser un écroulement uniquement si
$c(s) \geqslant \left|V^+(s)\right|$.
\end{definition}

\begin{definition}[Stabilité]
Une configuration $c$ est dite stable si
$\forall s \in S, c(s) \leqslant \left|V^+(s)\right|$.
\end{definition}

\begin{definition}
On montre que le résultat d'une suite d'écroulement est indépendant de l'ordre
dans lequel sont effectués les écroulements et qu'il existe toujours une suite
d'écroulements permettant d'obtenir une configuration stable.

Appelons avalanche $c^\circ$ le résultat d'une telle suite.
\end{definition}

\begin{definition}[Monoïdes des tas de sable]
Soit $G = (S, A, p)$ un graphe de tas de sable.
Un tas de sable sur ce graphe est un configuration stable.
Les tas de sable munis de la loi $+$ forment un monoïde.

Notons $???$ ce monoïde.
\end{definition}

\end{document}