\documentclass[french,11pt,a4paper]{article}
\usepackage[T1]{fontenc}
\usepackage{amsmath}
\usepackage{amsfonts}
\usepackage{amssymb}
\usepackage{amsthm}
\usepackage{babel}

\usepackage{dsfont} % for \mathds{1}

\usepackage{geometry}
\geometry{
	a4paper,
	top=2cm,
	bottom=2cm,
	right=3cm,
	left=3cm
}

\title{Brouillon lac}
\author{Hugo Vangilluwen}
\begin{document}
	\maketitle
	
	\begin{abstract}
		Le but de ce TIPE est de comprendre comment jet d'eau, mouvement rectiligne uniforme, induit sur un plan d'eau un mouvement répétitif, ondulatoire.
	\end{abstract}
	
	\vspace{1cm}
	
	\section{Définitions générales}
	
	Posons le lac ou plan d'eau dont appartiennent toutes les gouttes d'eau :
	\begin{equation}
		Lac = [0;1]^2
	\end{equation}
	En réalité, le lac pourrait être un ensemble quelconque.
	
	À chaque goutte d'eau, on associe une hauteur qui est l'écart par rapport à la position d'équilibre. Posons l'ensemble des hauteurs possible :
	\begin{equation}
		Hauteur = \mathcal{F} \left( Lac , \mathbb{R} \right)
	\end{equation}
	
	Chaque goutte d'eau ou plutôt chaque volume mésoscopique de surface du lac s'écoule sur les gouttes d'eau avoisinante. Notons l'influence $I$, c'est-à-dire la 'forme' des gouttes influencées. L'écoulement est une fonction suivante :
	\begin{equation}
		\acute{E} : \mathbb{R} \rightarrow \mathcal{F} \left( I , \mathbb{R} \right)
	\end{equation}
	telle que :
	\begin{equation}
		\tag{Conservation de la matière}
		\int_I \acute{E}(x, \mu) d\mu = \acute{E}(x, 0_I)
	\end{equation}
	$I$ pourrait être remplacé par $[0;\alpha]^2$ avec $\alpha \in [0;1]$.
	
	Définissons l'écoulement centré par
	\begin{equation}
		\acute{e} \ \left| \begin{matrix}
			Lac \times Hauteur &\rightarrow& Hauteur \\
			(g,h) &\mapsto& \left| \begin{matrix}
				Lac &\rightarrow& \mathbb{R} \\
				g' &\mapsto& \left\{ \begin{matrix}
					\acute{E}(h(g))(g') &\text{si}& g' \in g + I \\
					0_{\mathbb{R}} &\text{sinon}
				\end{matrix} \right.
			\end{matrix} \right.
		\end{matrix} \right.
	\end{equation}
	
	Maintenant, définissons une étape d'écoulement pour tout volume $v$ :
%	\begin{equation}
%		\begin{matrix}
%			Hauteur &\rightarrow& Hauteur \\
%			h &\mapsto& \left| \begin{matrix}
%				Lac &\rightarrow& \mathbb{R} \\
%				v &\mapsto& \displaystyle h(v) + \int_{Lac} \acute{e}(\nu, h)(v) \mathrm{d}\nu
%			\end{matrix} \right.
%		\end{matrix}
%	\end{equation}
	\begin{equation}
		\frac{dh}{dt}(v) = \int_{Lac} \acute{e}(\nu, h)(v) \mathrm{d}\nu
	\end{equation}
	
	\section{Simulation informatique}
	Pour la simulation, il est nécessaire de discrétiser les ensembles ci-dessus.
	\begin{equation*}
		\begin{aligned}
			Lac &= [\![ 0 ; n ]\!] \\
			I &= [\![ -m ; m ]\!] \quad \text{avec} \quad m \leqslant n \\
		\end{aligned}
	\end{equation*}
	Les intégrales deviennent des sommes. On utilise la méthode d'Euler pour pouvoir calculer une solution de l'équation fonctionnelle. \\
	Les valeurs pour les hauteurs sont limitées entre $-1$ et $1$.
\end{document}